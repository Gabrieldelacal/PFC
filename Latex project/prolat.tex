\documentclass[twoside]{article}
\usepackage{graphicx}

%\usepackage[latin1]{inputenc}
\usepackage[utf8]{inputenc}
\usepackage[spanish]{babel}
\usepackage{amsmath, amsthm, amsfonts}
\usepackage{textcomp}
\usepackage[T1]{fontenc}
\usepackage{hyperref}
\usepackage{array}
\usepackage{MnSymbol}   % Para poder escribir el cuadrado negro

\usepackage{geometry}
    \geometry{a4paper,total={210mm,297mm},left=35mm,right=30mm,top=30mm,bottom=30mm,}

\usepackage{tikz}
\usetikzlibrary{arrows,shapes,trees}

\title{\begin{center} 
\includegraphics[scale=0.3]{upc.jpg} 
\end{center} 
\vspace{1cm} 
Projecte Final de Carrera\\
ENGINYERIA INDUSTRIAL \\
\vspace{1.5cm} 
\Huge{Control d'un Quadcopter} 
\vspace{2cm} \\ 
Memòria}
%\author{Gabriel de la Cal Mendoza}
\date{}

\usepackage{fancyhdr}		%% Para cabezera de página

\pagestyle{fancy}

% \lhead[x1]{x2}
% \chead[y1]{y2}
% \rhead[LE,RO]{z2}
% \renewcommand{\headrulewidth}{0.5pt}

% \lfoot[a1]{b2}
% \cfoot[c1]{d2}
% \rfoot[LE,RO]{\includegraphics[scale=0.1]{etseib.jpg}}
% \renewcommand{\footrulewidth}{0.5pt}

\fancyhead[LO,RE]{Quadcopter Control}
\fancyhead[LE,RO]{\thepage}
\fancyfoot[LE,RO]{\includegraphics[scale=0.1]{etseib.jpg}}
\fancyfoot[C]{}

\begin{document}
\maketitle
\begin{center}
\large{
$\begin{array}{ll}
\mbox{Autor:} & \mbox{Gabriel de la Cal Mendoza} \\
\mbox{Director:} & \mbox{Manel Velasco Garcia} \\
\mbox{Convocatòria:} & \mbox{Data a presentar}
\end{array}$}
\\ \vspace{2cm} \Large{Escola Tècnica Superior d'Enginyeria Industrial de Barcelona}\\ \vspace{1cm}
\includegraphics[scale=0.4]{etseib.jpg}
\end{center}

\thispagestyle{empty}
\newpage
\begin{center}

\end{center}
\thispagestyle{empty}
\newpage
\setcounter{page}{1}
\section*{Resum}
\newpage
\begin{center}

\end{center}
\thispagestyle{empty}
\newpage

\setcounter{page}{1}
\tableofcontents
\addcontentsline{toc}{section}{Resum}
\fancyhead[LE,RO]{3}
\fancyfoot[C]{}
\newpage
\fancyhead[LE,RO]{\thepage}
\setcounter{page}{4}
\listoffigures
\newpage

\section{Prefaci} 
\subsection{Motivació}

La principal motivació d'aquest projecte és la d'aplicar els coneixements bàsics adquirits en aquests 6 anys de carrera, i més en particular en l'àrea del control en haver fet l'intensificació d'automàtica.\\

Llavors, a l'hora de plantejar un tema per al projecte ràpidament va sorgir la idea de realitzar-lo sobre el control d'un sistema, i més en particular sobre un que estigués actualment emergent tant en mercats com en el camp de l'investigació.\\ 

D'entre les diferents alternatives, un quadcopter és la més atractiva tant per la seva simplicitat constructiva com pel no excessiu cost. Existeixen actualment en el mercat infinitat de proveïdors dels components necessaris per a construïr un quadcopter, amb un gran ventall d'opcions per a escollir cada element com motors, bateries, electrònica, etc.\\

Les possibles aplicacions són nombroses, tant que encara no s'han ni tan sols explotat totes les possibles. Com a exemple: vigilància de superfícies obertes, transport de petits paquets, eina d'oci, entre d'altres.\\

%Com a experiència prèvia es va realitzat el Pre-Projecte sobre quadcopters amb la plataforma open source $ArduPilot$.   

El preu d'un quadcopter no es gaire elevat ja que un comercial com el $AR Drone 2.0$ oscil·la els 300 € i en construïr un amb menys prestacions no hauria de superar aquest cost.

\subsection{Requeriments previs}

Com a requeriments previs és necessari tenir certs coneixements mínims en automàtica per tal de controlar el quadcopter, aíxí com l'inquietut per aprendre el que sigui necessari per a complir, en la mesura del possible, els objectius inicials del present projecte com superar les dificultats que hagin sorgit, tenint en compte que el fi no justifica els medis.
 
\newpage
\section{Introducció}
Un Quadcopter es un vehicle volador no tripulat ($Unmanned Aerial Vehicle$) que es caracteritza per tenir quatre rotors a mode d'actuadors en comptes de dos com en el cas dels helicòpters. Aquest tipus d'autogir intenta obtenir una flotabilitat estable i vol precís balancejant les forces produïdes pels quatre motors. \\

Un dels avantatges que s'obtenen amb aquest canvi és la major capacitat de càrrega ja que es tenen 4 motors  per a soportar el pes. L'estabilitat del vehicle millora en permetre aterratges i enlairaments verticals amb una major maniobrabilitat. També pot treballar en àreas de difícil accés o més agressives, com amb pluja i vent. \\



\subsection{Estudi de l'art}
\subsection{Objectius del projecte}
Esquemàticament es pot representar com una estructura en $X$ amb el seu centre coincidint amb el centre de masses i quatre actuadors a les puntes de cada braç, tots ells apuntant en la mateixa direcció y sentit

descripción por encima\\
historia\\
estudio del arte\\
futuras aplicaciones\\

% \section{Estudi de l'art}
\newpage
\section{Definicio del model}
\subsection{Definició de les variables}
Per a caracteritzar la planta amb la que es treballarà, és necessari obtenir un model del quadcopter. Les constants pròpies del model es deixaran en forma de paràmetres a calibrar una vegada es tingui l'objecte físic. D'aquesta manera el model serà general per a tot quadcopter que comparteixi la mateixa família de paràmetres.
És necessari considerar dos marcs de referència: l'inercial format pels eixos $x,y,z$ i el del cos (Body) format pels eixos $x_B,y_B,z_B$. El primer té la persepectiva de l'observador en terra, estàtic, mentre que el segon és solidari a l'estructura. Segons l'orientació del eixos del cos amb aquesta referència es poden donar els següents dos cassos:
\begin{itemize}
\item $Cross type$: Els eixos de coordenades coincideixen amb els braços de l'estructura ja que es tenen els actuadors a les puntes de cada braç.
\item $X-type$: Els eixos i l'estructura formen 45º. Es tenen llavors dos motors al davant i dos al darrere.
\end{itemize} 
Per ser més usual la primera opció, es decideix utilitzar la configuració $Cross type$ tal i com es té en la figura \ref{RefQuad}.
\begin{figure}[h!]
\centering
\includegraphics[scale=0.5]{quad.jpg}
\caption{Marcs de referència en el quadcopter}
\label{RefQuad}
\end{figure}\\
Es suposa que l'objecte és un rotor esfèric, i per tant el seu tensor d'inèrcia és diagonal:
\begin{equation}I=\left[ \begin{array}{ccc}
I_{xx} & 0 & 0 \\
0 & I_{yy} & 0 \\
0 & 0 & I_{zz} 
\end{array} \right] \end{equation}
Es defineix la posició linear absoluta amb les coordenades $x,y,z$ pel vector $\xi$ i igualment per a la posició angular a partir de $\eta$ segons:
\begin{equation}
\xi=\left[ \begin{array}{ccc}
x\\
y\\
z \end{array} \right] ,\quad \eta=\left[ \begin{array}{ccc}
\phi\\
\theta\\
\psi \end{array} \right] , \quad q=\left[\begin{array}{c}
\xi\\
\eta \end{array} \right] 
\end{equation}

on $\phi$ és l'angle de capcineig (Pitch), $\theta$ és el de balanceig (Roll) i $\psi$ el de guiñada (Yaw).\\
Per a l'orientació angular entre els dos marcs es té un sistema de referència amb angles Tait Bryan, on la matriu de transformació és:
\begin{equation}
R=\left[\begin{array}{ccc}
C_\psi C_\theta & C_\psi S_\theta S_\phi - S_\psi C_\phi & C_\psi S_\theta C_\phi + S_\psi S_\phi \\
S_\psi C_\theta & S_\psi S_\theta S_\phi + C_\psi C_\phi & S_\psi S_\theta C_\phi - C_\psi C_\phi \\
-S_\theta & C_\theta C_\phi & C_\theta C_\phi 
\end{array}\right]
\end{equation}
amb $C_\phi=cos(\phi)$ i $S_\phi=sin(\phi)$.\newpage
Les velocitats lineals en el marc de referència del cos (Body Frame) es representen amb el vector $v_B$ y les velocitats angulars amb $\gamma$ segons:
\begin{equation}
v_B=\left[\begin{array}{c}
v_{x,b}\\
v_{y,b}\\
v_{z,b}\\
\end{array}\right] \hspace{1cm} \gamma=\left[\begin{array}{c}
p\\
n\\
r
\end{array} \right]
\end{equation}
En canvi, les velocitats en el marc de referència inercial (Inertial Frame)  es representen per $\dot{\eta}$ per a les velocitats lineals i per $\dot{\xi}$ per a les angulars: 
\begin{equation}
\dot{\xi}=\left[\begin{array}{c}
\dot{x} \\
\dot{y} \\
\dot{z}
\end{array} \right] \hspace{1cm} \dot{\eta}=\left[\begin{array}{c}
\dot{\psi} \\
\dot{\theta} \\
\dot{\psi}
\end{array} \right] 
\end{equation}
Ja que la derivada dels angles $\phi$, $\theta$ i $\psi$ no és el vector de velocitats angulars és necessari tenir un canvi de base per relacionar el marc de referència inercial al del cos amb la matriu $W_\eta$:
\begin{equation}
W_\eta=\left[\begin{array}{ccc}
1 & 0 & -S_\theta \\
0 & C_\phi & C_\theta S_\phi \\
0 & -S_\phi & C_\theta C_\phi 
\end{array} \right] \hspace{0.5cm} amb \hspace{0.5cm} 
\gamma=\left[ W_\eta \right] \dot{\eta} 
\end{equation}
Les forces de sustentació i moments generats pels quatre actuadors són $f_1,f_2,f_3,f_4$ i $w_1,w_2,w_3,w_4$ respectivament. Seguint l'orientació de la figura \ref{RefQuad}, per tal de poder anular els moments produïts en l'eix $z_B$ (en el marc del cos) el sentit de gir dels actuadors  $4$ i $2$ és en el de les agulles del rellotge (clockwise) i els dels $1$ i $3$ en sentit contrari (counterclockwise). \\

Interessa conèixer quina força i moment aportarà cada motor per a una velocitat angular coneguda. Es tenen les següents relacions per a cada actuador: 
\begin{equation}
\begin{array}{l}
f_i=kw^2_i \\ 
\tau_{M_i}=bw^2_i
\end{array}
\end{equation}
Per tant l'empenta total $T_B$ proporcionada en la direcció $z_B$ i els moments generats $\tau_B$ pels motors és:
\begin{equation}
T_B=k\left(\sum_{i=1}^{4}w^2_i \right)e_{z_B}=\left[ \begin{array}{c}
0 \\
0 \\
\displaystyle\sum_{i=1}^{4}w^2_i
\end{array} \right] 
\hspace{1cm} \tau_B=\left[ \begin{array}{c}
\tau_\phi \\
\tau_\theta \\
\tau_\psi
\end{array} \right] = \left[ \begin{array}{c}
lk(w^2_4 - w^2_2) \\
lk(w^2_3 - w^2_1) \\
\displaystyle\sum_{i=1}^{4}\tau_{M_i}
\end{array} \right]
\end{equation}

\subsection{Obtenció del model}
Les equacions que governen el sistema s'obtenen amb el mètode de Euler-Lagrange, pel que es parteix obtenint el Lagrangià del sistema:
\begin{equation}
\mathcal{L}=E_{cinetica} - E_{potencial} = (E_{translacio}+E_{rotacio})-E_{potencial}
\end{equation}

Substituïnt cada component per la seva expressió:
\begin{equation}
\mathcal{L}(q,\dot{q})=\frac{m}{2} \dot{\xi}^T \dot{\xi} + \frac{1}{2}\gamma^{T}I\gamma - mgz
\end{equation}
Es troba el vector de forçes i moments com:
\begin{equation}
F=\left[ \begin{array}{c}
f \\
\tau_B
\end{array} \right] = \frac{d}{dt}\left(\frac{\partial \mathcal{L}}{\partial \dot{q}}\right)-\frac{\partial\mathcal{L}}{\partial q}
\end{equation}

amb $ \hspace{0.5cm} q=[\begin{array}{cccccc}
x & y & z & \phi & \theta & \psi
\end{array} ]^{T} \hspace{0.5cm}$ i $ \hspace{0.5cm} \dot{q}=[\begin{array}{cccccc}
\dot{x} & \dot{y} & \dot{z} & \dot{\phi} & \dot{\theta} & \dot{\psi}
\end{array} ]^{T}$.\\

En calcular $F$ és necessari fer el canvi de variables de $\gamma$ a $\dot{\eta}$ amb el canvi de base $\dot{\eta}=\left[ W_\eta \right]^{-1} \gamma$ per tal de poder derivar el Lagrangià respecte $\dot{q}$, que són les variables pròpies del marc de referència inercial

\begin{equation}
\frac{1}{2}\gamma^{T}I\gamma = \frac{1}{2}(W_\eta \dot{\eta})^{T}I(W_\eta \dot{\eta}) = \frac{1}{2}\dot{\eta}^{T}(W_\eta ^{T}IW_\eta)\dot{\eta} = \frac{1}{2}\dot{\eta}^{T}J\dot{\eta}
\end{equation}

on la matriu $J$ queda com
\begin{equation}
J=W_\eta ^{T}IW_\eta = \left[ \begin{array}{ccc}
I_{xx} & 0 & -I_{xx} S_\theta \\
0 & I_{yy} C^2_\phi + I_{zz} S^2_\phi & (I_{yy}-I_{zz}) C_{\phi} S_\phi C_\theta \\
-I_{xx} S_\theta & (I_{yy}-I_{zz}) C_{\phi} S_\phi C_\theta & I_{xx} S^2_{\theta}+I_{yy} S^2_{\phi} C^2_\theta +I_{zz}C^2_{\phi} C^2_{\theta}
\end{array} \right]
\end{equation} 

i per tant el Lagrangià queda com:

\begin{equation}
\mathcal{L}(q,\dot{q})=\frac{m}{2} \dot{\xi}^T \dot{\xi} + \frac{1}{2}\dot{\eta}^{T}J\dot{\eta}- mgz
\end{equation}
Les components lineals y angulars no depenen unes de les altres, i per tant es poden estudiar per separat obtenint dos equacions: una per les forces lineals y un altre per als moments. Això vol dir que la força que exerceixen els actuadors no depenen de les velocitats angulars que es tinguéssin, i tampoc es tindran accel·leracions angulars diferents segons l'altura a la que es trobi el quadcopter: l'objecte girarà de la mateixa manera sigui quina sigui la seva posició en l'espai.

Llavors, fent la derivada parcial respecte $\dot{q}$ s'obté
\begin{equation}
 F=\frac{d}{dt}\left(\frac{m}{2}(1\cdot\dot{\xi}+\dot{\xi}\cdot 1)+\frac{1}{2}\frac{\partial}{\partial \dot{\eta}}(\dot{\eta}^{T}J\dot{\eta})\right)-\frac{\partial \mathcal{L}}{\partial q}
\end{equation}

Com que $J$ és una matriux simètrica, es pot dir que  $\frac{\partial}{\partial \dot{\eta}}(\dot{\eta}^{T}J\dot{\eta})=2 \frac{\partial}{\partial \dot{\eta}}(\dot{\eta}^{T}J)\dot{\eta} $. \\

\textbf{Demostració:} Per probar això es veurà per al cas $\frac{\partial}{\partial x}(x^{T}Ax)=2 \frac{\partial}{\partial x}(x^{T}A)x $ amb 

\begin{equation}
x=\left[ \begin{array}{c}
x_1 \\
x_2 \\
x_3 
\end{array} \right] \hspace{1cm} A=\left[ \begin{array}{ccc}
a_{11} & a_{12} & a_{13} \\
a_{21} & a_{22} & a_{23} \\
a_{31} & a_{32} & a_{33} 
\end{array} \right]
\end{equation}
Llavors
\begin{equation}
\frac{\partial}{\partial x}(x^{T}A x) = \frac{\partial}{\partial x}\left(\left[x_1 x_2 x_3 \right]\left[\begin{array}{ccc}
a_{11} & a_{12} & a_{13} \\
a_{21} & a_{22} & a_{23} \\
a_{31} & a_{32} & a_{33} 
\end{array} \right]\left[\begin{array}{c}
x_1 \\
x_2 \\
x_3 
\end{array} \right] \right) =
\end{equation}
\begin{equation}
\frac{\partial}{\partial x}(x^2_1 a_{11}+x_1x_2a_{21}+x_1x_3a_{31} + x_1x_2a_{12}+x^2_xa_{22}+x_2x_3a_{32} + x_1x_3a_{13}+x_2x_3a_{23}+x^2_3a_{33})=
\end{equation}
Com que $A$ és simètrica $a_{12}=a_{21}$, $a_{13}=a_{31}$ i $a_{23}=a_{32}$, i en fer la derivada direccional resulta
\begin{equation}
\frac{\partial}{\partial x}(x^{T}A x)=\left[ \begin{array}{c}
2x_1a_{11}+x_2a_{21}+x_3a_{31}+x_2a_{12}+x_3a_{13} \\
x_1a_{21}+x_1a_{12}+2x_2a_{22}+x_3a_{32}+x_3a_{23} \\
x_1a_{31}+x_2a_{32}+x_1a_{31}+x_2a_{23}+2x_3a_{23}
\end{array} \right]=2\cdot\left[ \begin{array}{c}
x_1a_{11}+x_2a_{12}+x_3a_{13} \\
x_1a_{12}+x_2a_{22}+x_3a_{23} \\
x_1a_{13}+x_2a_{23}+x_3a_{23}
\end{array} \right]
\end{equation}
I en avaluar l'altre costat de la igualtat es té el mateix resultat
\begin{equation}
2 \frac{\partial}{\partial x}(x^{T}A)x=2 \frac{\partial}{\partial x} \left( \left[ \begin{array}{ccc}
x_1a_{11}+x_2a_{12}+x_3a_{13} \\
x_1a_{21}+x_2a_{22}+x_3a_{23} \\
x_1a_{13}+x_2a_{32}+x_3a_{33}
\end{array} \right]^{T} \right)x=
\end{equation}
\begin{equation}
=2\left[\begin{array}{ccc}
a_{11} & a_{12} & a_{13} \\
a_{21} & a_{22} & a_{23} \\
a_{31} & a_{32} & a_{33} 
\end{array} \right]\cdot\left[\begin{array}{c}
x_1 \\
x_2 \\
x_3
\end{array} \right]=2\cdot\left[ \begin{array}{c}
x_1a_{11}+x_2a_{12}+x_3a_{13} \\
x_1a_{12}+x_2a_{22}+x_3a_{23} \\
x_1a_{13}+x_2a_{23}+x_3a_{23}
\end{array} \right]
\end{equation}
\hfill $\blacksquare$ \newpage
Com que $2 \frac{\partial}{\partial \dot{\eta}}(\dot{\eta}^{T}J)\dot{\eta}=2 J \dot{\eta}$ es té, aplicant la regla de la cadena en el producte $J\dot{\eta}$:
\begin{equation}
F=\frac{d}{dt}\left(m \dot{\xi}+J \dot{\eta}\right)-\frac{\partial \mathcal{L}}{\partial q} =m\ddot{\xi} + J\ddot{\eta} + \dot{J}\dot{\eta}-\left( \frac{1}{2} 2 \frac{\partial}{\partial \dot{\eta}} ( \dot{\eta}^{T}J)\dot{\eta} - mg\left[\begin{array}{c}
0 \\
0 \\
1
\end{array} \right] \right)
\end{equation}

Per arribar a aquest resultat s'ha aplicat la derivada direccional a $mgz$:
\begin{equation}
D_q(mgz)=D_\xi(mgz)=mg \left[ \begin{array}{c}
0 \\
0 \\
1
\end{array} \right]
\end{equation}

Separant les components lineals i angulars en dues equacions:
% \begin{equation}
\begin{align}
f & = m \ddot{\xi} + mg \left[ \begin{array}{c}
0 \\
0 \\
1
\end{array} \right] =RT_B\\
\tau & =J \ddot{\eta} +  \underbrace{\left( \dot{J} - \frac{\partial}{\partial \dot{\eta}}(\dot{\eta}^{T}J)\right)}_{C(\eta,\dot{\eta})} \dot{\eta} =J \ddot{\eta} +  C(\eta,\dot{\eta})\dot{\eta} 
\end{align}
% \end{equation}

On $C(\eta,\dot{\eta})$ és la matriu de Coriolis. 
Per obtenir el sistema d'equacions del model s'han d'aïllar les accel·leracions, i s'obté:
\begin{equation}
\begin{cases}
\ddot{\xi} & =\frac{1}{m}RT_{B} - g \cdot \left[ \begin{array}{ccc}
0 & 0 & 1\\
\end{array} \right]^{T} \\
\ddot{\eta} & =J^{-1} \left( \tau - C(\eta,\dot{\eta})\dot{\eta}  \right)
\end{cases}
\label{eq:system}
\end{equation}

Per a realitzar el control será útil representar el sistema \ref{eq:system} en forma d'espai d'estats:

\begin{equation}
\frac{\partial}{\partial t}\left[ \begin{array}{l}
\xi \\
\dot{\xi} \\
\eta \\
\dot{\eta} 
\end{array} \right] =\left[ \begin{array}{l}
\dot{\xi} \\
\frac{1}{m}RT_{B} - g \cdot \left[ \begin{array}{ccc}
0 & 0 & 1 \\
\end{array} \right]^{T} \\
\dot{\eta} \\
J^{-1} \left( \tau - C(\eta,\dot{\eta})\dot{\eta} \right)
\end{array} \right]
\end{equation}

\subsection{Representació del model amb Matlab}





\newpage
\section{Disseny del controlador}
% Explicar que se quiere controlar el Quadcopter mediante la Linealización extendida, trabajando alrededor de puntos de equilibrio
% \Linealització extesa
% Explicar el concepto de Linealización extendida 
% Explicar el proceso de obtención del lagrangiano con el .m,  vector de fuerzas, obtención del sistema linealizado alrededor del punto de equilibrio, observador lineal,... 
\newpage
\section{Implementació del control} 
% Pufff

\newpage
\section{Construcció del Quadcopter}
% Descripción de cómo se hará en general
El conjunt de peces que formen aquest aparell estan conectades entre sí segons la funció que realitzen. Esencialment la Raspberry Pi controla els motors segons les señals que reb de l'IMU i el Receptor. Tot el conjunt és alimentat per una bateria LiPo i s'adapta el voltatge de 11.1V a 5V per mitjà d'un Regulador per tal d'alimentar a la Raspberry. Tots els components estan subjectats a una estructura (Frame) que també pateix les forces y moments.  

\subsection{Descripció dels components}
% Descripción y explicación de cada componente del Quadcopter, con imágenes de cada
Es descriu tot seguit cada component i el criteri de sel·lecció que s'ha aplicat.
\subsubsection*{Raspberry Pi} 
Abreujat com a RPi, és un petit ordinador integrat en una sola placa (Single-Board Computer o SBC en anglès) del tamany d'una targeta de crèdit, és a dir, amb unes dimensions de 85.6cm x 53.98cm, desenvolupat per la Fundació Raspberry Pi amb l'intenció de promocionar les ciències computacionals a les escoles \cite{RPiWiki}. 

Es un petit ordinador integrat en una sola placa (Single-Board Computer o SBC en anglès) del tamany d'una targeta de crèdit, és a dir, amb unes dimensions de 85.6cm x 53.98cm, desenvolupat per la Fundació Raspberry Pi amb l'intenció de promocionar les ciències computacionals a les escoles \cite{RPiWiki}. 

S'ha optat per aquesta opció pel seu econòmic preu, la velocitat de processament i baix consum. A més, s'ha volgut ampliar els coneixements d'aquest petit monstre. En particular s'utilitza la segona revisió del model B:
 
\begin{tabular}{cc}
\hspace{1cm}\includegraphics[scale=0.06]{RPi.jpg} & \hspace{0.5cm} \includegraphics[scale=0.3]{RPi2.jpg}
\end{tabular} \\

Té un System-on-Chip (SoC) Broadcom BCM2835 amb un ARM1176JZF-S a 700 Mhz, una GPU VideoCore IV i 512 MB de memòria RAM. Disposa de dos ports USB, una sortida mini-jack 3.5mm, sortida d'audio/vídeo HDMI, una sortida RCA i un port RJ45 10/100 d'Ethernet. 

L'alimentació es realitza per mitjà d'un mini USB a 5V/700mA, amb un consum de 3.5W. El sistema operatiu és un Raspbian, gravat en una targeta SD de 4GB. 

Disposa d'un conjunt de pins que permeten comunicació amb perifèrics de baix nivell UART, I2C, SPI i 8 pins de propòsit general (General Porpouse Input Output o GPIO).

\subsubsection*{GY-521 MPU-6050}
\begin{table}[!h]
\begin{tabular}{m{1.5cm}m{12.5cm}}
\hspace{1cm}\includegraphics[scale=0.1]{mpu-6050.jpg} & Es tracta d'una Unitat de Mesura Inercial (IMU en anglès) que integra en un mateix encapsulat de 4x4x0.9mm un acceleròmetre i un giròscop, ambdós de 3 eixos. Disposa dun conversor ADC de 16 bits per a cada eix i es comunica mitjançant un protocol de comunicació I2C a $400kHz$. S'ha optat per utilitzar aquest dispositiu pel seu baix cost i la fàcil comunicació que comporta amb la RPi.\\
\end{tabular}
\end{table}
Com a característiques dels sensors: el giròscop té un rang de  $\pm250,\pm500,\pm1000,\pm2000$ graus/segon, i l'accel·leròmetre de $\pm2g,\pm4g,\pm8g,16g$. La tensió d'alimentació és del rang de 2.375V-3.46V.

\subsubsection*{Transmissor-Receptor}
Amb aquest parell de components es transmet la consigna generada des del transmissor cap al receptor. El model que s'utilitza és el Turnigy 5X 5Ch Mini, per què és un model fàcil d'utilitzar i 
\\
\begin{table}[!h]
\begin{tabular}{m{7cm}m{12cm}}
\includegraphics[scale=0.4]{E_R.jpg} & \begin{itemize}
\item asdf 
\end{itemize}
\end{tabular}
\end{table}

5 Channel radio control
• Secure 2.4GHz FHSS
• Dual rates
• Analog trim tabs
• Servo reverse
• Switchable mode setting (mode-1-mode-2)
• Fixed wing and delta mixes
• LED indicator light
• Adjustable stick length
• Folding Antenna


\subsubsection*{Emissor Receptor}

\subsubsection*{Bateria LiPo}

\subsubsection*{Regulador Step-Down}
\subsubsection*{Variadors ESC}
Turnigy AE-20A Brushless ESC
Specification:
Output:  Continuous 20A, burst 25A up to 10 seconds.
Input Voltage:  2-4 cells lithium battery or 5-12 cells NIMH battery.   
BEC:  Linear 2A @ 5V
Control Signal Transmission: Optically coupled system.
Max Speed:     
   2 Pole: 210,000rpm
   6 Pole: 70,000rpm
   12 Pole: 35,000rpm
Size:  50mm (L) * 26mm (W) * 12mm (H).
Weight:  19g.

Features:
High performance microprocessor brings out the best compatibility with all kinds of motors and the highest driving efficiency.
Wide-open heatsink design to get the best heat dissipation effect.
Improved Normal, Soft, Very-Soft start modes, compatible with aircraft and helicopter. 
Smooth, linear, quick and precise throttle response.
Multiple protection features: Low-voltage cut-off protection / Over-heat protection / Throttle signal loss protection
Programable via transmitter
Programming features:
Brake setting (we recommend using brake for only folding props applications)
Battery type(Li-xx or Ni-xx) 
Low voltage cutoff setting 
Factory default setup restore 
Timing settings (to enhance ESC efficiency and smoothness) 
Soft acceleration start ups (for delicate gearbox applications)
Low voltage cutoff type (power reduction orirnmediate shutdown)
 
Factory default settings:
Brake:  off 
Battery type:  Li-xx (Li-ion or Li-Po) 
Low voltage cutoff threshold:  Soft cut-off (2.6V) 
Timing setup:  Low 
Soft Acceleration Start Up:  Normal  
Low voltage cutoff type:  Medium
\subsubsection*{Motors //modelo}
Turnigy 2213 20turn 1050kv 19A Outrunner
Spec.
Kv: 1050rpm/v
Operating Current: 6A ~ 16A
Peak Current: 19A
Weight: 56g
Dimensions: 27.6 x 32mm
Shaft Size: 3.175mm
\subsubsection*{Frame}
Peso 190g
\subsubsection*{Hèlix}

\newpage
\subsection{Muntatge}
% Descripción del procedimiento por pasos y con fotos

\begin{center}
\begin{tikzpicture}
%%% POWER LINES
\path[draw] (1,8.25) node {\large 12V};
\path[draw] (2,8.25) node {\large 0V};
\path[draw,line width=6pt,color=red,o-o] (1,1) -- (1,8);
\path[draw,line width=6pt,color=black,o-o] (2,1) -- (2,8);

%%% LIPO LEVEL
\path[draw,line width=2pt,color=red] (1,6) -- (3,6);
\path[draw,line width=2pt,color=black] (2,5.5) -- (3,5.5);
\path[draw] (3.8,5.75) node[draw,line width=3pt] (nodeA) {\huge LiPo};
\path[draw] (7.2, 5.75) node[draw,line width=3pt] (nodeA) {\huge IMU};
\path[draw,line width=2pt,color=olive] (7.2,5.35) -- (7.2,4.6);
\path[draw] (7.6,5) node (nodeC) { I2C};
\path[draw] (9.8, 5.75) node[draw,line width=3pt] (nodeA) {\huge Rec.};
\path[draw, line width=2pt,color=green] (9.8, 5.34) -- (9.8, 4.5) -- (7.95, 4.5);
\path[draw] (9, 4.8) node {5xCHNL};
\path[draw,line width=2pt,dotted, color=cyan](10.6,5.75) -- (12.3,5.75);
\path[draw] (13, 5.75) node[draw,line width=3pt] (nodeA) {\huge Em.};

%%% BUCK LEVEL
\path[draw,line width=2pt,color=red] (1,4.5) -- (3,4.5);
\path[draw,line width=2pt,color=black] (2,4) -- (3,4);
\path[draw] (4.1,4.25) node[draw,line width=3pt] (nodeA) {\huge BUCK};
\path[draw,line width=2pt,color=red] (5.26,4.5) -- (6.5,4.5);
\path[draw,line width=2pt,color=black] (5.2,4) -- (6.5,4);
\path[draw] (7.2,4.25) node[draw,line width=3pt] (nodeA) {\huge RPi};
\path[draw] (5.8,4.8) node (nodeC) {5V};
\path[draw, line width=2pt, color=green] (7.95,4)--(9.8,4) -- (9.8,3.2);
\path[draw] (9.2,3.65) node {PWM};

%%% ESC LEVEL
\path[draw, line width=2pt, color=red] (1,3) -- (9,3);
\path[draw, line width=2pt, color=black] (2,2.5) -- (9,2.5);
\path[draw] (9.8,2.75) node[draw,line width=3pt] (nodeC) {\huge ESC};
\path[draw] (10.3,2.1) node (nodeC) {\large (x4)};
\path[draw, line width=2pt, color=blue] (10.6,3) -- (12,3);
\path[draw, line width=2pt, color=blue] (10.6,2.75) -- (12,2.75);
\path[draw, line width=2pt, color=blue] (10.6,2.5) -- (12,2.5);
\path[draw] (13,2.75) node[draw,line width=3pt] (nodeC) {\huge Motor};
\path[draw] (13.7,2.1) node (nodeC) {\large (x4)};
\end{tikzpicture}
\end{center}

% \section{Generació de la consigna}
% Descripcion general de como se genera, envia y procesa la consigna
% Imagen con esquema
% \subsection{Configuració de la RPi com a Access Point}
% Describir procedimiento de configuración de la RPi para ser AP
% \subsection{Aplicació d'Android }
% Explicar que la App es en java y para Android y que envia la informacion a la RPi (configurada como AccessPoint).
% Dar el codigo de la aplicacion

% \section{Comparació amb un control PID}

\newpage
\section{Anàlisi econòmic}

\newpage
\section*{ANNEXES}
\subsection*{Annex 1: Obtenció vector de forces}
asdfasdf 

\subsection*{Annex 2:}

\subsection*{Annex 2:Preparació de la Raspberry Pi}
Per a posar a punt la RPi s'ha d'instal·lar Raspbian i configurar-lo per tal que es pugui comunicar amb la placa MPU-6050.
\subsubsection*{Instalació Raspbian}
Havent introduït la targeta SD en un lector adeqüat, es detecta en una terminal mitjançant l'ordre \textit{df -h}. Suposant que hagués estat gravada anteriorment:
\begin{center}
\includegraphics[scale=0.7]{InstalRasp1.jpeg}
\end{center}
S'han de desmontar les dos particions, tant la \textit{/dev/sdb1} com la \textit{/dev/sdb2}:
\begin{verbatim}
           umount /dev/sdb1
           umount /dev/sdb2
\end{verbatim}
Havent descarregat la imatge a instalar per exemple a l'Escriptori, es procedeix amb:
\begin{verbatim}
    sudo dd bs=4M if=~/Escritorio/2013-07-26-wheezy-raspbian.img of=/dev/sdb
\end{verbatim}
Passats uns minuts ja es té la SD grabada amb el sistema operatiu Raspbian.
\subsubsection*{Configuració de la Raspberry}
Per tal de poder comunicar la RPi amb la IMU MPU-6050 per I2C és necessari configurar el sistema. Primer és necessari instal·lar els drivers més rellevants. De l'arxiu
\begin{verbatim}
           sudo nano /etc/modules
\end{verbatim}
s'han d'afegir les següents dues línies al final de l'arxiu:
\begin{verbatim}
           i2c-bcm2708
           i2c-dev
\end{verbatim}
En l'arxiu blacklist:
\begin{verbatim}
           sudo vi /etc/modprobe.d/raspi-blacklist.conf
\end{verbatim}
les següents dues línies han de començar amb un signe \# (de comentari):
\begin{verbatim}
           #blacklist spi-bcm2708
           #blacklist i2c-bcm2708
\end{verbatim}
Per provar de connectar el sensor, és necessari seguir primer el connexionat:
\begin{center}
\includegraphics[scale=0.2,viewport=0 200 800 720,clip]{ConSens.jpg}
\end{center}
És a dir:
\begin{itemize}
\item Pin1-3.3V es connecta VCC.
\item Pin3-SDA es connecta a SDA
\item Pin5-SCL es connecta a SCL
\item Pin6-Ground es connecta a GND
\end{itemize} 
S'ha d'instal·lar el paquet $i2c-tools$:
\begin{verbatim}
           sudo apt-get install i2c-tools
\end{verbatim}
Per a veure el sensor s'escriu:
\begin{verbatim}
           sudo i2cdetect -y 1
\end{verbatim}
El resultat és 
\begin{center}
\includegraphics[scale=0.7]{i2cdetect.jpg}
\end{center}
que és el dispositiu que correspon al MPU-6050.

\newpage
\begin{thebibliography}{99}
\bibitem{QuadPaper} Teppo Luukkonen (August 22,2011). \textit{Modelling and control of quadcopter}
\bibitem{RPiWiki} Wikipedia de la Raspberry Pi: \url{http://en.wikipedia.org/wiki/Raspberry_Pi}
\bibitem{IMUArduino} Accel·leròmetre i Giròscop MPU-6050 per a Arduino: \url{http://playground.arduino.cc/Main/MPU-6050#.UzhsVCK9jb4}
\bibitem{6050Esp} MPU-6050. Especificació del producte: \url{http://www.invensense.com/mems/gyro/documents/PS-MPU-6000A-00v3.4.pdf}
\bibitem{6050RegMap} MPU-6050. Mapa de Registres i descripcions: \url{http://www.invensense.com/mems/gyro/documents/RM-MPU-6000A-00v4.2.pdf}
\end{thebibliography}{}
\end{document}